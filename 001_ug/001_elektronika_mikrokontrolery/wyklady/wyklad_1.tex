\documentclass{article}

\usepackage[utf8]{inputenc}
\usepackage[T1]{fontenc}
\usepackage{verbatim}
\usepackage{graphicx}

\title{Podstawy Elektroniki i Programowania Mikrokontrolerów}

\begin{document}

\maketitle

\newpage \tableofcontents

\section{Podstawowe informacje o prądzie
elektrycznym}

\subsection{Ładunki i pola}
Oddziaływanie elektromagnetyczne należy do jednego z czterech fundamentalnych oddziaływań w fizyce. 
\subsubsection{Ładunek elektryczny}
\textbf{Ładunek elektryczny} należy traktować jako dodatkową cechę materii, która pozwala mu na oddziaływanie z innymi ładunkami poprzez pole elektromagnetyczne. 
\begin{itemize}
    \item Ładunek może mieć dwa przeciwne znaki: dodatni (+) i ujemny (-) decydujące o zwrocie oddziaływań
    \item Ładunek dodatni - proton
    \item Ładunek ujemny - elektron
    \item Ładunek jest skwantowany, to znaczy może występować tylko w ściśle określonych porcjach będących wielokrotnością ładunku elementarnego elektronu e.
    \item Jednostka ładunku - Kulomb (C) - ładunek elektryczny przenoszony przez prąd o natężeniu 1A w czasie 1s
    \item Całkowity ładunek elektryczny układu zamkniętego nie ulega zmianie
\end{itemize}

\subsubsection{Pole elektryczne}
\textbf{Pole elektryczne} to pole wektorowe określające w każdym punkcie \textbf{siłę działającą na jednostkowy, spoczywający ładunek elektryczny}. Pole elektryczne, niezależnie od układu ładunków, które je wytworzyły, można całkowicie zdefiniować, podając wartość natężenia pola $E(x, y, z)$ w każdym punkcie $(x, y, z)$ rozpatrywanego obszaru przestrzeni.

Umieszczenie ładunku \(q\) w polu \(\vec{E}\) powoduje powstanie siły $\vec{F}$

\textbf{Natężenie pola elektrycznego}: $\vec{E} = \frac{\vec{F}}{q}$

\subsubsection{Potencjał pola elektrycznego}
\textbf{Potencjał pola elektrycznego} w punkcie to stosunek pracy $W$ potrzebnej do przeniesienia ładunku $q$ z tego punktu do nieskończoności.
$$V = \frac{W}{q}$$
Jednostka potencjału jest 1 V (wolt) równy 1 J / 1 C (dżulowi na kulomb).

\subsection{Prąd elektryczny}
\textbf{Prąd elektryczny} to uporządkowany ruch ładunków elektrycznych. Ilościowo prąd elektryczny najprościej określa się przez ustalenie wielkości ładunku przepływającego w jednostce czasu przez jakąś wyodrębnioną powierzchnię. Tak zdefiniowana wielkość nosi nazwę natężenia prądu elektrycznego, a jej jednostka jest amper (1 A).
\begin{itemize}
\item Natężenie prądu stacjonarnego, czyli niezmiennego w czasie lub wielkość średnia natężenia prądu: $I=\frac{q}{t}$
\item Natężenie prądu jako funkcja czasu: $i(t)=\frac{d}{dt}q(t)$
\end{itemize}

\subsection{Napiecie elektryczne}
\textbf{Różnica potencjałów elektrycznych między dwoma punktami obwodu elektrycznego} lub pola elektrycznego. Napięcie elektryczne jest to stosunek pracy wykonanej przeciwko polu, podczas przenoszenia ładunku elektrycznego między punktami, dla których określa się napięcie, do wartości tego ładunku.
$$U_{AB} = V_B - V_A = \frac{W_{A \rightarrow B}}{q}$$
Jednostka napiecia jest V (Wolt)(Patrz: potencjał elektryczny).

\subsection{Praca i moc}
Praca to miara ilości energii przekazywanej między układami fizycznymi.
\begin{itemize}
\item $W = U \cdot I \cdot t = q \cdot U$
\item Jej jednostką jest J (dżul).
\end{itemize}

Moc to wielkość określająca pracę wykonaną w jednostce czasu przez układ fizyczny.
\begin{itemize}
    \item $P=\frac{W}{t} = I \cdot U = I^2 \cdot R = \frac{U^2}{R}$
    \item Jej jednostką jest W (Wat).
\end{itemize}

\subsection{Prawo Ohma}
Prawo fizyki głoszące \textbf{proporcjonalność natężenia prądu płynącego przez przewodnik do napięcia panującego między końcami przewodnika}.
$$I(t) \sim U(t)$$
$$i(t) = \frac{1}{R}\cdot u(t)$$
$$I=\frac{U}{R}$$

\begin{itemize}
    \item $R$ to \textbf{rezystancja} (opór). Jej jednostką jest $\Omega$ (Ohm).
    \item Elementy, które spełniają prawo Ohma to \textbf{rezystory}. Nie wszystkie elementy elektroniczne spełniają prawo Ohma.
\end{itemize}

\subsection{Siła elektromotoryczna}
Gdyby do jednego końca przewodu podłączyć naładowane elektrycznie ciało, a do drugiego identyczne, ale z niedoborem ładunku, to prąd przez przewodnik płynąłby tylko do czasu osiągnięcia równowagi przy ciągłym zmniejszaniu się napięcia na jego końcach. Zatem \textbf{do utrzymania stałej różnicy potencjałów potrzebne są źródła napięcia}. SEM to \textbf{zdolność źródła do wytwarzania różnicy potencjałów}. Jest mierzona w woltach i można ją traktować jako idealną różnicę potencjałów, którą źródło wytwarza, by wprowadzić elektrony w ruch.

\subsection{Prawa Kirchhoffa}
\subsubsection{Prawo Kirchhoffa I (prądowe)}
Suma natężeń prądów wpływających do węzła jest równa sumie natężeń prądów wypływających z tego węzła.
\subsubsection{Prawo Kirchhoffa II (napięciowe)}
W zamkniętym obwodzie (oczku) suma spadków napięć równa jest sumie sił elektromotorycznych występujących w tym obwodzie.

\subsection{Źródła}
\subsubsection{Źródło napięciowe}
Źródło dające stałą SEM (napięcie) niezależnie od obciążenia.
\begin{itemize}
    \item Źródło idealne - nie uwzględnia np. rezystancji wewnętrznej
    \item Źródło rzeczywiste - bateria, zasilacz
\end{itemize}

\subsubsection{Źródło prądowe}
Źródło dające stały prąd niezależnie od obciążenia.
\begin{itemize}
    \item Źródło idealne - nie uwzględnia np. rezystancji wewnętrznej
    \item Źródło rzeczywiste - akumulator samochodowy
\end{itemize}

\subsection{Laczenie rezystencji}
\begin{itemize}
    \item Polaczenie szeregowe: $R=\sum R_i$
    \item Polaczenie rownolegle: $R=\sum \frac{1}{R_i}$
\end{itemize}

\subsection{Kondensator}
Uklad modulujacy pole elektryczne.
$$u(t) = \frac{1}{C}\cdot\int_0^{t}i(t)\cdot dt + u(0)$$
$$i(t)=C\cdot\frac{du(t)}{dt}$$

Pojemnosc: C. Jednostka: Farad (F)

\subsection{Indukcyjnosc}
$$u(t)=L\cdot\frac{di(t)}{dt}$$
$$i(t)=\frac{1}{L}\cdot\int_0^t u(t)\cdot dt + i(0)$$

Jednostka: henr (H), pojemnosc: L

\end{document}